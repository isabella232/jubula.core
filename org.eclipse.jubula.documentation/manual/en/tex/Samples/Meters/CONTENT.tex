The meters application is a piece of software designed to manage meter readings for different flats in a building complex. The building complex is displayed on the left hand side in a tree, which contains nodes showing the flats and their meters. 

The most basic use case for this application is to take a meter reading. Taking a meter reading is also a part of the other two use cases shown in this example, moving a tenant in and moving a tenant out of the complex. 

This test uses central test data sets to manage its data \bxpref{TasksCentralData}. 

If you do not have write privileges in the \app{} installation directory, you must create a workspace directory for the meters application and enter this workspace in the \gdaut{} configuration for meters. In the \bxname{AUT Arguments} field, enter: \bxshell{-data <path to workspace>}.

%% \bxname{/examples/Samples.zip}\\
%% You will need to take the following steps to be able to run the meters tests:
%% \begin{enumerate}
%% \item Open the \app{} Workspace Perspective:\\
%% \bxmenu{Window}{Open Perspective}{\app{}Workspace}
%% \item Right-click in the \gdnavview{} and select:\\
%% \bxmenu{Import}{}{}\\
%% from the context-sensitive menu. 
%% \item In the dialog that appears, select \bxname{Existing Projects into Workspace} from the \bxname{General} category. 
%% \item Press \bxcaption{Next}.
%% \item In the next dialog, activate the \bxname{Select archive file} option.
%% \item Browse to the \app{} installation directory and then to the
%% \bxname{examples} directory. 
%% \item Select the \bxname{Samples.zip}. 
%% \item Click \bxcaption{Finish} in the dialog. The Excel files for the meters tests will appear in the \gdnavview{}. 
%% \item If you do not have write privileges in the \app{} installation directory, you must also create a workspace for the meters application and enter this workspace in the \gdaut{} configuration for meters. In the \bxname{AUT Arguments} field, enter: \bxshell{-data <path to workspace>}.
%% \end{enumerate}


\subsection{Sample 5: Tests with the Meters tool}
\index{GEF}
Like other tests, tests for GEF components are based on the component, action, parameter principle. However, it is important to note that individual figures are not addressed using component names. The only component that is mapped in GEF is the \bxname{figure canvas}.  The figures on the canvas are tested by referring to their textpath. See the Reference Manual for more details on the GEF actions. 

\bxtipp{The observation mode cannot be used to record actions in the canvas.}



