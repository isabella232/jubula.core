% $Id: accessing.tex 10532 2010-03-16 15:57:20Z alexandra $
% Local Variables:
% ispell-check-comments: nil
% Local IspellDict: american
% End:
% --------------------------------------------------------
% User documentation
% copyright by BREDEX GmbH 2004
% --------------------------------------------------------
You can find the \gdproject{} containing the tests
in the installation directory under the subdirectory:\\
\bxname{examples/projects}. 

You can import the \gdproject{} into the \ite{} as follows:
\begin{enumerate}
\item Select:\\ \bxmenu{Test}{Import}{}.
\item Browse to the \bxname{examples/projects} directory in the installation directory. 
\item Select both the \bxname{samples} \gdproject{} and the \bxname{bound\_modules\_samples} \gdproject{}. 
\item  Select \bxcaption{OK} in the \bxname{Import Project} dialog. 
\end{enumerate}


Once the \gdprojects have been imported, open the \bxname{samples} \gdproject{}. Start and connect to the \gdagent \bxpref{Agent}, and start the \gdaut{} \bxpref{startaut}. 


When working on one machine, it may be a good idea to automatically minimize 
 the \ite{} during test execution. This can be done via :\\
\bxmenu{Window}{Preferences}{}. \\
In the \bxname{Test} preferences. 
Alternatively, the client window can be reduced so 
that both the client and the \gdaut{} can be seen. 

\subsubsection{Result Reports}

A result report of the test will be automatically saved into the \gddb{} when the test has run. The summary of the reports can be viewed in the \gdtestsummaryview{} \bxpref{testresultview}.
