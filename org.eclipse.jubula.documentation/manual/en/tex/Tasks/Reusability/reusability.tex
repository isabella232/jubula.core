% $Id: reusability.tex 4493 2007-02-08 09:58:18Z mike $
% Local Variables:
% ispell-check-comments: nil
% Local IspellDict: american
% End:
% --------------------------------------------------------
% User documentation
% copyright by BREDEX GmbH 2005
% --------------------------------------------------------
% this command can be inserted multiple times
%\gdhelpid{}
% 
%\begin{gddescription}
%5\end{gddescription}
%
%\begin{gdlist}
% use the \item command for single steps
%5\end{gdlist}
% change <PATH> to the same directory, file is located in
% change <FILE> to the same filename you are editing
%\bxinput{<PATH>/Links/<FILE>}
%
% other usefull commands are
%   \bxtipp{}        to create a hint%   \bxwarn{}        to describe a warning
\index{Reusability}
\label{reusability}
\begin{itemize}
\item There are two methods of reusing \gdcases{}:
\begin{enumerate}
\item Adding a \gdcase{} to a \gdsuite{}
\item Nesting a \gdcase{} in another \gdcase{}. 
\end{enumerate}
\item When a \gdcase{} is reused, it displays a small arrow. 
\gdmarpar{../../../share/PS/testCaseRef}{reused \gdcase{}}
\item A \gdcase{} can also be used as an \gdehandler{}. For more information on this, see the section later \bxpref{Eventhandling1}.
\end{itemize}


\subsection{Adding \gdcases{} to a \gdsuite{}}
See the previous section for information on adding \gdcases{} to a \gdsuite{} \bxpref{TasksEditorAdd}.


\subsection{Nesting one \gdcase in another}
See the previous section for information on adding \gdcases{} to a \gdcase{} \bxpref{TasksEditorAdd}.


