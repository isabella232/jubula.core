\index{Model Diagram!Import}
\index{Import!Model Diagram}
You can import models directly into \jb{} which have been created with the Eclipse UML2 Tools. The UML2 tools project is a sub-project of the Model Development Tools project. 

In this project, models are always saved in two files - \bxname{*.umlusc} and \bxname{*.uml}. The \jb{} import dialog does not show the *.uml files, but it is important that these *.uml files are in the same folder as the *.umlusc files. 



\begin{enumerate}
\item In the \gdnavview{},  select:\\
\bxmenu{Import}{}{}\\
from the context-sensitive menu. 
\item In the wizard that appears, select \bxname{Import external Model File} from the \bxname{Model} category. 
\item Click \bxcaption{Next}. 
\item Enter of browse to the model diagram you want to import (the *.umlusc file). 
\item Select or enter the \gdproject{} you want to import the diagram into. 
\item Click \bxcaption{Finish}. The model diagram you specified  will be imported into your workspace as a \jb{} model digram. It appears in the \gdnavview{} in the \gdproject{} you specified. 
\item The following elements are imported from the diagram:
\begin{itemize}
\item Use cases
\item Relations between use cases (\bxname{include} relationships)
\item Packages
\end{itemize}

Now you have imported a model, you can make changes to it \bxpref{TasksMBTModel}, validate it \bxpref{TasksMBTValidate} or generate it \bxpref{TasksMBTGenerate}. 
\end{enumerate}

