%% % $Id: placeholders.tex 2243 2006-06-12 13:50:53Z alexandra $
%% % Local Variables:
%% % ispell-check-comments: nil
%% % Local IspellDict: american
%% % End:
%% % --------------------------------------------------------
%% % User documentation
%% % copyright by BREDEX GmbH 2005
%% % --------------------------------------------------------
%% % this command can be inserted multiple times
%% \gdhelpid{}
%% % 
%% \begin{bxdescription}
%% \end{bxdescription}
%% %
%% \begin{bxsteps}
%% % use the \item command for single steps
%% \end{bxsteps}
%% % change <FILE> to the same filename you are editing
%% \bxinput{Links/<FILE>}
%% %
%% % other usefull commands are
%% %   \bxhint{}        to create a hint
%% %   \bxwarn{}        to describe a warning


\index{Test Case!Empty}
\index{Empty Test Case}
\index{Top-Down Specification}
\index{Specification!Approaches!Top-Down}
\begin{itemize}
\item If it helps you to structure your test, you can leave \gdcases{} empty.
\item You can then add \gdsteps{} and \gdcases{} later and/or use the \gdcase{} as an \gdehandler{}.
\item In this way, you can develop the structure of a test before you know exactly what the individual steps will be. 
\item Empty \gdcases{} in a \gdsuite{} do not disrupt the execution, they are simply marked as ''successful'' in the \gdtestresultview{}. 
\end{itemize}

 

