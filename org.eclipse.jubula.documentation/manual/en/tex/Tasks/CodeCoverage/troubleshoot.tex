When using code coverage, please bear the following in mind:

\begin{itemize}
\item Code Coverage is only possible with  \gdauts{} started via an \gdaut{} configuration \bxpref{configuringaut} (i.e. not using the \bxname{autrun} command), and which use Java 1.5 or higher.
\item JaCoCo manipulates the byte code of your \gdaut{} at runtime to be able to measure code coverage. It is therefore highly sensitive to other byte code manipulations that take place at the same time (e.g. cglib).
\item If you wish to analyze the Code Coverage at the source code level, as well as entering the \bxname{Source Directory} in the \gdaut{} configuration, you must also ensure that your class files have been compiled with debug information.
\item Running Code Coverage analyses can become memory-intensive for larger \gdauts{}. A pattern can be used to reduce the code analyzed. You can also increase the heap space for the \ite{} to ensure that enough memory is available. We have successfully performed code coverage analysis with JaCoCo on an \gdaut{} with 72,000 classes.  
\item Users working with the embedded \gddb{} may run into memory problems sooner than users working with an Oracle \gddb{}. Please remember that we do not recommend working with the embedded \gddb{} for productive use. 
\end{itemize}


\textbf{The \gdaut{} does not start when Code Coverage is enabled}\\
If the \gdaut{} starts normally without Code Coverage but does not when Code Coverage is activated, ensure that you are not using the embedded \gdagent{}.

\textbf{The \gdtestsummaryview{} displays the monitoring agent as \bxname{JaCoCo}, but the coverage value is 0}\\
If you can open the HTML report for the Code Coverage, but it shows \bxcaption{NaN}, then this could mean that the class files for the analysis were not found. Check the path to the \bxname{Installation Directory} in the \gdaut{} configuration. 
It could also be the case that other byte code manipulations were running at the same time as JaCoCo.






