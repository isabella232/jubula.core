You can activate code coverage for an \gdaut{} configuration with the following steps:

\begin{enumerate}
\item Open the \gdaut{} configuration dialog from the \gdproject{} properties \bxpref{projectproperties}.
\item Select the \bxname{Expert} configuration. 
\item Select the code coverage agent you wish to use (\bxname{JaCoCo} is available out-of-the-box). 
\item You can then enter the AUT installation directory and the AUT source directory for the code coverage:
\begin{description}
\item [The AUT installation directory]{ is the directory containing the class files (compiled Java files) for your \gdaut{}. You must enter this directory to make code coverage possible for your test run.}
\item [The AUT source directory]{ is the directory where the source files (i.e. the program code) for your \gdaut{} are kept. Entering a directory for the source files is optional, however, if you do not enter one, then you will not be able to view your code coverage results at the source file level. The AUT source directory must contain the source files in their Java package structure. The class files must have been compiled with debug information to make the lines of code executed visible in the code coverage report.}
\bxtipp{You can enter relative paths for the AUT installation and AUT source directories. The paths are relative to the working directory.}
\end{description}
\item To make sure you only monitor your own code, enter a package pattern to specify which packages should be monitored. The pattern must be a valid regular expression. If you do not enter a package pattern, all classes in the virtual machine will be considered for the code coverage value.
\bxwarn{Not entering a package pattern can result in extremely large messages being sent from the \gdagent{}, which may cause memory problems \bxpref{CCHeap}.}
\item Select whether you want the code coverage value to be reset when a new \gdsuite{} starts \bxpref{CCReset}.  
\end{enumerate}


\subsubsection{Increasing the Java Heap Space for code coverage}
\label{CCHeap}
Running a test with a code coverage profiler leads to an increased memory requirement for the \ite{}. You can increase the heap space for the \ite{} and also enter a package pattern \bxpref{CCConfigure} to reduce the amount of files considered for code coverage.  

\bxtipp{Users working with a MySQL \gddb{} should also follow the steps from the Installation Manual to increase the maximum allowed packet for the \gddb{}. }
