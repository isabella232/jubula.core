To make sure you get the correct results for your code coverage, it is important to be aware of which actions will accumulate the recorded code coverage and which actions will reset the value back to zero. 


\textbf{Code coverage reset}\\

The following result in the code coverage value being reset:

\begin{itemize}
\item Stopping the \gdaut{}. When the \gdaut{} is started again, the code coverage value is reset to zero. The test executor ensures that all monitoring information is collected before stopping the \gdaut{}. 
\end{itemize}

\bxwarn{If you stop the \gdaut{} while code coverage is being calculated, then the value will also be set to zero!}

\textbf{Code coverage accumulation}\\
The following result in code coverage information being accumulated:

\begin{itemize}
\item Any actions you perform manually in an \gdaut{} that has been started with code coverage will contribute to the code coverage result.
\item Code coverage is accumulated across \gdsuites{} by default. If you want to reset the code coverage at the beginning of each \gdsuite{}, then select this option in the \gdaut{} configuration \bxpref{CCConfigure}.
\bxtipp{If you are working with \gdjobs{}, then do not opt to reset the code coverage at the beginning of a \gdsuite{}.}
\item If you use the \bxname{restart} action during a test, this does \textbf{not} result in the code coverage value being reset.
\end{itemize}

