

Once you have created a \gdproject{}, you can define (and edit) \gdauts{}. 
You can define a new \gdaut{} straight after creating the \gdproject{} in the \gdproject{} wizard or you can do it later on via the \gdproject{} properties \bxpref{projectproperties}. 

\bxtipp{If you will be starting your \gdaut{} with the \bxname{autrun} command \bxpref{autrun}, then you can automatically define your \gdaut{} \bxpref{createAUTDef}}

The \gdaut{} dialog (\bxfigref{autdialog}) appears when you define or edit an \gdaut{}. 

\begin{figure}[h]
\begin{center}
\includegraphics[width=12.5cm]{Tasks/AUTs/PS/autdialog}
\caption{AUT Dialog}
\label{autdialog}
\end{center}
\end{figure}

\bxtipp{If you know that you will be working with multiple versions of the same \gdaut{} (e.g. a version for Windows and Linux, or two versions that use different databases), then define \bxname{one} \gdaut{} here and create multiple configurations \bxpref{configuringaut} for this \gdaut{}. This means that your different configurations will all share one object map. If you are working with multiple completely different \gdauts{}, then define the different \gdauts{} here.}

\begin{enumerate}
\item Enter a meaningful and unique \gdaut{} name. This is used to easily identify the \gdaut{} later. 
\item Select the toolkit the \gdaut{} uses from the combo box. 
\item If you choose RCP, decide whether or not \app{} should generate names for components in the \gdaut{} which have not been named by your developers \bxpref{RCPgenerate}. We recommend leaving this option checked, as it increases the robustness of your tests. 
\item If you are starting a Java \gdaut{} and  will be starting it using the \bxname{autrun} command \bxpref{autrun}, or if the \gdaut{} will be launched from another \gdaut{} during the test \bxpref{TasksLenientTest}, then enter the ID(s) for these \gdauts{} here. 
\textbf{IDs for \gdauts{} started by the \bxname{autrun} command}\\
Enter the \gdaut{} IDs you will use for any \gdauts{} started by the \bxname{autrun} command (the \gdaut{} ID for the \gdaut{} is given as a parameter in the \bxname{autrun} command \bxpref{autrun}.

\textbf{IDs for \gdauts{} launched by other \gdauts{}}\\
The \gdaut{} ID will take a specific form \bxpref{TasksLenientTest} and must be defined as such in the \gdaut{} definition.
 
\bxtipp{If you will be starting your \gdaut{} from \app{} (i.e. via an \gdaut{} configuration \bxpref{configuringaut}) then you do not need to enter any IDs here.}
\item From the list of \gdproject{} languages, select which languages are supported by the \gdaut{}. 

The languages you select are the languages the \gdaut{} can be started in.  You will be able to translate the data in your \gdcases{} into these languages so that a \gdsuite{} will test the \gdaut{} in the right language. 

\bxwarn{If you are editing the \gdaut{} and remove an  \gdaut{} language for which you have already specified data, this will result in the data for that language being lost. }

\item If you want to start this \gdaut{} via \app{}, you can do so in the \gdproject{} properties \bxpref{configuringaut}.
\end{enumerate}

If you do not require an \gdaut{} configuration, because you will be starting the \gdaut{} using the \bxname{autrun} command \bxpref{autrun}, then you do not need to create an \gdaut{} configuration. 

\clearpage
