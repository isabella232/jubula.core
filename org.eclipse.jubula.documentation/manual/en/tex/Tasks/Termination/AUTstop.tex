
% $Id: AUTstop.tex 10592 2010-03-19 15:52:43Z alexandra $
% Local Variables:
% ispell-check-comments: nil
% Local IspellDict: american
% End:
% --------------------------------------------------------
% User documentation
% copyright by BREDEX GmbH 2004
% --------------------------------------------------------
\index{AUT!Stop}
\index{Stop!AUT}
\begin{enumerate}
\item When you stop the \gdagent{}, any \gdauts{} that were connected to this \gdagent{} are closed.
\item \gdauts{} which were started from a configuration are automatically closed when the test execution via the test executor stops.
\item To stop the \gdaut{} manually, select the \gdaut{} you want to stop from the \gdrunautview{} and then click the {Stop \gdaut{}} button. 
\gdmarpar{../../../share/PS/stopAUT}{ stop \gdaut{}}
\item A dialog will appear to ask you if you are sure you want to stop the \gdaut{}. Click \bxcaption{yes}. 
\item You can opt to not see this dialog in the  preferences \bxpref{gdprefs}. 
\end{enumerate}

\textbf{Stopping an \gdaut{} in a test}\\
Stopping an \gdaut{} as part of a test is generally not possible. The \gdaut{} will be stopped automatically at the end of testing according to the points mentioned above.

You can use the \bxname{restart} action to restart the \gdaut{} during the test. You may be able to use the action \bxname{External Key Combination} to send an unconfirmed keystroke to the \gdaut{} (i.e. \bxkey{ENTER} on a prompt dialog for closing the \gdaut{}) to close the \gdaut{} at the end of your test, however, the success of this is dependent on the \gdaut{} and environment as well as timing and synchronization issues. 
