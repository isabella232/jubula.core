Although the \dash{} is primarily intended to be started on a central machine with access to a central \gddb{} in which the test results reside, it is also possible to start the \dash{} in a default configuration to allow access to the embedded \gddb{}. This option is useful if you are evaluating the function and want to use the \dash{} on your local machine without having to enter any configuration information. 

\begin{enumerate}
\item In the installation directory, open the \bxname{dashboard} folder. 
\item In this folder, start the \bxname{dashboard} application (Windows users use \bxname{dashboard.exe}, Unix users use \bxname{dashboard.sh}.  
\item Starting this application starts the \bxname{dashboardserver} with default parameters for the embedded database. You can see which parameters are used by looking in the \bxname{dashboardserver.properties} file which is in the same folder as the \dash{} application. 
\item While the server is starting, you will see a progress window. A system tray icon for the \dash{} server will appear on your screen. 
\item Once the server is started, a browser opens containing the \reportpersp{} in which you can view and open Test Result Reports from the \gddb{} \bxpref{TasksDashUse}. 
\bxwarn{Bear in mind that if you are not using the embedded \gddb{}, then you will have to start the \dash{} server with other parameters to connect to the \gddb{} you are using \bxpref{TasksDashCustom}.}
\end{enumerate}
