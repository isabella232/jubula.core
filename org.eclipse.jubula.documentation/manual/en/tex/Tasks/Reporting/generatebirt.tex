From the \gdtestsummaryview{} \bxpref{TestResSumView}, you can generate reports of your test runs over time using the BIRT reporting engine. 

\bxtipp{Only test runs that have been marked as relevant are included in the generated BIRT reports. You can change the relevance from the \gdtestsummaryview{} \bxpref{TasksChangeRelevance}.}

\begin{itemize}
\item The \ite{} offers a selection of example reports:
\begin{description}
\item [Test Comments:]{This report shows a table of all failed tests for the time chosen, whose relevance is set to true and whose name does not include \bxname{BROKEN}. The comment title that can be entered in the \gdtestsummaryview{} \bxpref{TestSummaryComments} is also shown. This report is useful for delivering daily status reports of the tests.}
\item [Test Execution Overview:]{This report shows a list of executed, relevant \gdsuites{} over the time period selected. For each day in the time period, there is a colored block for the \gdsuite{} to display its status. Green means that the test ran without any errors. Red means that the test ran, but with errors. Yellow means that the test ran more than once on that day. White means that the \gdsuite{} did not run on this day. This report is useful for teams with large amounts of \gdsuites{} who need to ensure on a daily basis that all \gdsuites{} were started. \gdsuites{} that were not run at all during the selected time period are not displayed.}
\item [Test Duration:]{This report shows the duration of the chosen tests.}
\item [Test Execution Histogram:]{This report shows the proportion of executed, failed and non-executed \gdsteps{} for a test. This report is most useful when one specific \gdsuite{} is compared to see its progress over time. }
\item [Test History:]{This report shows a graph of the percentage completion for the selected \gdsuites{} for the dates given. There is also a list of the \gdauts{}, \gdsuites{} and test runs.}
\item [Test History Absolute:]{This report shows the same details as the Test History report, but instead of showing the test results in percentages, shows the actual amount of \gdsteps{} executed. It also shows the difference between expected and executed \gdsteps{}.}
\item [Test History Absolute and Coverage:]{This report shows the same details as the Test History Absolute report with any code coverage information that is available for the chosen test runs.}
\item [Testresult:]{This report gives out the full test details up to the given nesting level.}
\item [Testresult Summary:]{This report shows a table of the \gdtestsummaryview{} for the dates and tests chosen. }
\end{description}

\item To start a report, click the arrow next to the \bxcaption{Create Report} button and select the report you want to generate.
\gdmarpar{../../../share/PS/createBirtReport}{create BIRT report}

\bxtipp{If you are not already connected to the \gddb{}, then a dialog will appear to create the connection \bxpref{tasksdblogin}. }

\item The BIRT report viewer starts (the first time it starts it may take some time).
\item The parameters for the report are displayed:
\begin{description}
\item [Selection:]{In this section, specify which time frame the report should be generated for (either from a specific date, or using the options \bxname{yesterday, now, last week} etc.) as well as for which \gdproject{}, \gdsuite{}, \gdjob{} and the operating system.  The details in the selection section are combined using \bxname{and}. SQL syntax can be used (e.g. \verb+%+ is used as a wildcard for any number of any characters, \verb+_+ is used as a wildcard for one character). }
\item [Test run ID:]{Some reports are generated for a specific test run. The test run ID of the currently selected test summary is entered by default, but you can enter a different run.}
\item [Detail Selection:]{For the Testresult and TestresultError reports, you must also specify the nesting level (how many levels in the hierarchy should be shown) and the ID of the test run you want to generate the report for. The test run ID for each run can be seen in the \gdtestsummaryview{}.}
\end{description}

\item Click \bxcaption{OK} to start the report generation. 
\item Once the report is ready, it will be shown in the BIRT viewer.
\item Hover over  the points on the graph in a report to see any additional information about the point (comment, data value etc.). 
\item You can click through the report's pages in the viewer and also export the report as a PDF. 
\bxwarn{The option \bxname{Auto} when exporting a report leads to the right-hand side of the report being cut off.}

\bxtipp{You can also create your own reports to execute \bxpref{OwnBIRT}.}
\end{itemize}
