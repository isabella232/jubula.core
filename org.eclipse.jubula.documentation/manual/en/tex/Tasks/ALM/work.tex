Once you have configured a task repository for your workspace \bxpref{TasksALMConfigureWorkspace}, you can work on tasks from that repository. 

\subsubsection{Opening and editing tasks in the \ite{}}
\begin{itemize}
\item To be able to see tasks in a repository, select: \\ \bxmenu{Window}{Show View}{Other}\\ from the menu. 
\item In the \bxname{Mylyn} section, select \bxname{Task List} and click \bxcaption{OK}. The \bxname{Task List} View will appear.
\item Double-click on a task to open this task in the editor area.
\item Once a task is open, you can work on it as you would in an external system -- add comments, change status etc.
\end{itemize}

\subsubsection{Working on tasks in the \ite{}}
\label{TasksActivateTask}
Mylyn supports context- or task-based working. When you work on a task, you only see items relevant to that task, so that coming back to the task later involves less context-switching. 
\begin{itemize}
\item Mylyn supports context-based working. You can work on existing tasks in a configured repository, or you can create tasks to work on.
\item To work on a task, you must \bxname{activate} it. To activate a task, select the task in the \bxname{Task List} and select:\\ \bxmenu{Activate}{}{}\\
from the context-sensitive menu. 
\item When you activate a task for the first time, the browsers and editors will seem very empty. This is because nothing is yet a part of the context for this task.
\item You can navigate through the browsers by pressing \bxkey{Alt+Click} to expand each level, or you can press the \bxname{Focus on task} button in the browsers to show the whole tree (not focusing on the task), or just the items in the current context (focusing on the task). 
\item Items are automatically added to your context when you select them in a browser, when you open them in an editor, or when you perform other actions that cause them to be made relevant (e.g. \gdcase{} creation, showing a \gdcase{} specification etc.). Items that are used particularly frequently are marked as \bxname{landmarks} and shown in bold. 
\item You can manually alter which items are in your context using the context-sensitive menu for a specific item. You can manually make items landmarks, or remove them from the context. 
\item The context that is created for you will be re-created when you reactivate the task at a later point. 
\end{itemize}

