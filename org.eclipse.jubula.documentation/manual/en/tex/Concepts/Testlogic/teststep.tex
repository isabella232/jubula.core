% $Id: teststep.tex 8161 2009-04-06 14:07:39Z alexandra $
% Local Variables:
% ispell-check-comments: nil
% Local IspellDict: american
% End:
% --------------------------------------------------------
% User documentation
% copyright by BREDEX GmbH 2004
% --------------------------------------------------------
\index{Test Step}
\index{Component}
\index{Parameter}
\index{Action}
\index{CAP}
\label{teststeps}

A \gdstep{} is the smallest unit in the test hierarchy. Each \gdstep{} represents one action on one component (or user-interface element) in the  \gdaut{}. 

The interaction is composed of three details, which we refer to as \bxcaption{CAP} (component, action, parameter):

\begin{description}
\item [Component:]{a component is a user-interface object (e.g. a button, a combo box).}
\item [Action:]{the action is the operation to be executed on the selected component. Each component has a number of actions which can be executed on it, for example, buttons can be clicked, an input can be made into a text field. }
\item [Parameter:]{the parameter(s) are the data or variables associated with an action. When a button is clicked, the parameter is the amount of clicks. When you are entering text into a text field, the parameter is the text you want to enter. The amount and type of parameters depends on the action.}
\end{description}

The only detail which needs to be fixed at this point in the specification is the action. The actual component to be tested and the parameters can be added or changed later on. 

The specification is also separate from the \gdaut{}. You give the component you specify a \bxname{component name}, which you use to identify the component in your test. This component name is assigned to the actual component in the \gdaut{} at a later point. In this way, specification can begin before the \gdaut{} is available. 

An example if  a \gdstep{} could be entering text (e.g. \bxshell{hello}) into a text field:
\\
\\
\begin{tabular}{p{0.3\bxpicwidth}p{0.3\bxpicwidth}}
 \textbf{Component}& Text field/text area/text pane/... \\
 \textbf{Action} & Enter text\\
 \textbf{Parameter} & hello\\
\\
\end{tabular}

\bxtipp{We recommend using the \gdcases{} in the library \gdprojects{} installed with \app{} instead of writing \gdsteps{}. Using the \gdcases{} from the library \gdprojects{} saves time and improves the flexibility of your tests.}
