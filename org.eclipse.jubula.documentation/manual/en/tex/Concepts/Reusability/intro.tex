%% % BREDEX LaTeX Template
%% %  \documentclass is either ``bxreport'' or ``bxarticle''
%% %                 option is bxpaper
%% \documentclass{bxarticle}
%% % ----------------------------------------------------------------------
%% \begin{document}
%% \title{}
%% \author{}
%% % \author*{Hauptautor}{Liste der Nebenautoren}
%% \maketitle
%% % ----------------------------------------------------------------------
%% \bxversion{0.1}
%% %\bxdocinfo{STATUS}{freigegeben durch}{freigegeben am}{Verteilerliste}
%% \bxdocinfo{DRAFT}{}{}{}
%% % ----------------------------------------------------------------------

%% \end{document}
\jb{} lets you create tests by reusing \gdcases{}. Once you have created a \gdcase{} in \jb{}, you can \bxname{reuse} it by adding it to other \gdcases{} and \gdsuites{}. 

The more general a \gdcase{} is, the easier it is to reuse in a variety of different places. The more you reuse \gdcases{} in your \gdproject{}, the easier it is to maintain your tests. 

\jb{} offers various features to improve the reusability of \gdcases{}. These features save time, avoid redundancy and make the tests you create well-structured and flexible. Instead of writing separate \gdcases{} for similar test situations, you can reuse a general \gdcase{}, and change the data or the component it tests. 


