% $Id: datasets.tex 2559 2006-07-26 08:35:50Z alexandra $
% Local Variables:
% ispell-check-comments: nil
% Local IspellDict: american
% End:
% --------------------------------------------------------
% User documentation
% copyright by BREDEX GmbH 2004
% --------------------------------------------------------
\index{View!Data Sets}
\index{Data Sets View}
\label{datasetsconcepts}

You can enter \emph{data sets} for parameters you have moved up to a \gdcase{}. Once you have used references in the \gdsteps{} or \gdcases{} nested in a  \gdcase{}, these parameters become a part of this parent \gdcase{}. You can then enter one or more data sets for these parameters. The whole \gdcase{} will be executed once for each data set, allowing you to test more data without specifying a \gdcase{} for each set.  

As with ordinary concrete parameter values, you can overwrite values in the \gddatasetsview{} when you reuse the \gdcase{}. 