% $Id: terms.tex 2559 2006-07-26 08:35:50Z alexandra $
% Local Variables:
% ispell-check-comments: nil
% Local IspellDict: american
% End:
% --------------------------------------------------------
% User documentation
% copyright by BREDEX GmbH 2004
% --------------------------------------------------------
\index{Parameter!Values}
\index{Parameter}
\index{References}
\index{Propagation}
\index{Foundation}
When we are talking about reusability, we will use the following terms:

\begin{description}
\item[Specification]{refers to the creation of test elements, most importantly \gdcases{}. A specified \gdcase{} is a \emph{master template}.}
\item[Reuse]{is the term used to describe the usage of a \gdcase{} in another \gdcase{} (nesting) or in a \gdsuite{}. }
\item[Overwriting]{is the process by which certain details or data are changed when a \gdcase{} is reused. Data which have been overwritten in a reused \gdcase{} will not change if the data in the master template change. }
\item[Parameters]{are the variables associated with a \gdstep{}.}
\item [Parameter values]{are the actual concrete integers or other inputs 
which will be used when the test is executed.}
\item [References]{are placeholders for parameters. They allow concrete 
parameter values to be specified at the next level (i.e. in the parent \gdcase{}. We refer to this as ''moving the parameter up''. References are preceded by 
the reference symbol (default is $=$). }
\item[Foundation]{refers to the names of references which make up a 
\gdcase{}. The names of references in any nested 
\gdcases{} also count as the foundation. These details may not be altered, since this would lead to 
instability.}
\end{description}

