% $Id: editors.tex 12243 2010-09-21 13:40:10Z alexandra $
% Local Variables:
% ispell-check-comments: nil
% Local IspellDict: american
% End:
% --------------------------------------------------------
% User documentation
% copyright by BREDEX GmbH 2004
% --------------------------------------------------------
Editors appear in the editor area in the middle of the perspective when you double-click an item  in one of the browsers or in the \gdnavview{}. 

You can tell what sort of editor it is in two ways:
\begin{enumerate}
\item The tab for each editor shows an ID-code for what type of editor it is:
\begin{itemize}
\item \bxcaption{TC} for the \gdtestcaseeditor{}
\item \bxcaption{TS} for the \gdtestsuiteeditor{}
\item \bxcaption{OM} for the \gdomeditor{}
\item \bxcaption{TD} for the \gddataeditor{}
\end{itemize}
The tab also shows the name of the item the editor is for. 

\end{enumerate}
 
 


\subsection{\gdtestcaseeditor{}}
\gdhelpid{guidancerSpecTestCaseEditorContextId}{Test Case Editor}
\index{Test Case!Editor}
 \index{Editor!Test Case}
You can open the \gdtestcaseeditor{} by double-clicking a \gdcase{} in the \gdtestcasebrowser{}.
 
In the \gdtestcaseeditor{}:
\begin{itemize}
\item You can create, modify and delete \gdsteps{}. 
\item You can add and delete other \gdcases{}. 
\item You can add \gdehandlers{}. They appear in the \gdehandler area below the \gdtestcaseeditor{}.  
\item You can alter test data and component names for the \gdcase{}.
\end{itemize}

When you single-click items (\gdcases{}, \gdsteps{}) in the \gdtestcaseeditor{}, the support views (\gdpropview{}, \gdcompnamesview{} and \gddatasetsview{}) show details about this item. Using these views, you can edit the properties, test data and component names for this item. 

\subsection{\gdtestsuiteeditor}
\gdhelpid{testSuiteEditorContextId}{Test Suite Editor}
\index{Test Suite!Editor}
\index{Editor!Test Suite}

You can open the \gdtestsuiteeditor{} by double-clicking a \gdsuite{} in the \gdtestsuitebrowser{}.


In the \gdtestsuiteeditor{}:
\begin{itemize}
\item You can add \gdcases{} to \gdsuites{}. 
\item You can alter the \gdsuite{} properties (e.g. the \gdaut{} it uses, the default \gdehandlers{}).
\item You can alter test data for the \gdcases{} shown. 
 \end{itemize}
 
\bxtipp{In the \gdtestsuiteeditor{}, you can only alter details for the top-level \gdcases{} for each subtree.}


 When you single-click \gdcases{} in the \gdtestsuiteeditor{}, the support views (\gdpropview{}, \gdcompnamesview{} and \gddatasetsview{}) show details about this \gdcase{}. Using these views, you can edit the properties, test data and component names for this item. 

\subsection{\gdomeditor}
\gdhelpid{objectMapEditorContextId}{Object Mapping}
\index{Object Mapping!Editor}
\index{Editor!Object Mapping}

Each \gdaut{} in a \gdproject{} has an \gdomeditor{}. You can open the  \gdomeditor{} by right-clicking on a \gdsuite{} in the \gdtestsuitebrowser{} and selecting \bxcaption{open with \gdomeditor{}}. 

You can also open the \gdomeditor{} by single-clicking the chosen \gdsuite{} and clicking on the \bxcaption{start \gdomm{}} button on the toolbar. This also starts the \gdomm{}. 

The \gdomeditor{} has four tabs, a split view, a tree view, a table view and configuration tab. 
In the \gdomeditor{}:
\begin{itemize}
\item You can see all the component names you have used in the \gdsuite{} you selected, and any other \gdsuites{} which use the same \gdaut{}. 
\item You can see any technical names you have collected from the \gdaut{}. 
\item You can see the component and technical names you have assigned to each other to complete mapping. 
\end{itemize}

\subsection{\gddataeditor{}}
\gdhelpid{guidancerCentralTestDataEditorContextId}{Central Test Data}
\index{Test data!Editor}
 \index{Editor!Central test data}
\index{Central test data!Editor}

You can open the \gddataeditor{} via the button on the toolbar. 

In the \gddataeditor{}:
\begin{itemize}
\item You can create, modify and delete central test data sets. 
\item You can search for places where central test data sets have been used. 
\end{itemize}


