You should name your \gdcases{} so that it is clear from the name what the \gdcase{} does. Instead of vague or ambiguous names (e.g. Select), try and write names that describe the \gdcase{} in terms of the action and the component it deals with. A few good examples are:

\begin{itemize}
\item Close Search Dialog with OK Button
\item Select Project Language from Combo Box
\item Fill in Project Wizard, containing:
\begin{itemize}
\item Enter Project Name in Wizard
\item Select Project Language from List
\item Select Default Language from Combo Box
\item Close Project Wizard with Finish Button
\end{itemize}
\end{itemize}

The \gdcase{} names do not have to exhaustively describe the component, or where it is in the \gdaut{} -- using categories for the different areas of your \gdaut{} saves you having to do this \bxpref{BPCategories}. The aim with naming your \gdcases{} is to make your tests readable and easy to understand. 

You do not have to write \gdcase{} names in CamelCase or with underscores -- \gdcase{} names are not used in the test executor, so there is no problem if they have spaces in them. 
