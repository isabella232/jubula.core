% $Id: SimpleMatch.tex 7708 2009-01-20 15:50:45Z alexandra $
% Local Variables:
% ispell-check-comments: nil
% Local IspellDict: american
% End:
% --------------------------------------------------------
% User documentation
% copyright by BREDEX GmbH 2005
% --------------------------------------------------------
\index{Simple Match}


\begin{itemize}
\item Using \bxname{simple match} lets you verify or select text even if you do not know exactly what the text is.
\item Using a system of placeholders for characters and some function symbols, you can enter a simple match expression to find or check a text. The patterns for simple match are similar to Unix-style globs or Windows wildcards. The exact syntax is defined below.
\item Actions which support regular expressions have an additional parameter, \bxcaption{Operator}. From this combo box, you can choose \bxcaption{simple match} to indicate that you want to use simple matching.
\end{itemize}

\textbf{Simple matching}
\begin{itemize}
\item \bxshell{abc} matches \bxcaption{abc} and nothing else.
\end{itemize}

\textbf{Wildcards}
\begin{itemize}
\item \bxshell{?} represents one instance of any character.
\item \bxshell{*} represents any number (zero or more) of any characters.  
\end{itemize}

\bxtipp{\app{} applies your expression to your entire string. To search for a match within a string, wildcards need to be placed on either side. See the examples below for more information.}


\bxtipp{The following syntax information can be useful, but is also more complex. We recommend that simple match expressions use only letters (upper- and lowercase), numbers, and the wildcards described above. In this way, you can avoid complex syntax.}
\textbf{Ranges}
\begin{itemize}
\item For each individual character, you can specify a range of things it is allowed to be. 
\item A range is specified using square brackets (\bxshell{[]}) and a dash \bxshell{-}. 
\item For example, you can specify that a particular character can be any capital letter: \bxshell{[A-Z]}.
\end{itemize}

\textbf{Escape character}
\begin{itemize}
\item Backslash \bxshell{$\backslash$} is used to negate the effect of the character following the backslash.
\item The characters that are used to construct a simple match expression need to be escaped if they are to be matched within a string.
\item The characters are: \newline
 [ ? * \newline 
\item Because \app{} already uses a backslash as an escape symbol, you will need to use two backslashes to escape simple match characters. 
\item For example, to check for a tree node \bxcaption{x/y/z/***} where the slashes are a part of the node, your regular expression in \app{}  would look like this: \verb1 x\/y\/z\/\\*\\*\\* 1\newline
The backslashes before the ordinary slashes are an escape symbol to tell \app{} that the following sign is not a path separator. The extra backslash before the stars tells \app{} that the second backslash is to be interpreted as a backslash in the regular expression, i.e. as an escape symbol.  
\end{itemize}

\textbf{Useful examples}
\begin{itemize}
\item A string that starts with \bxshell{a} is represented by: \bxshell{a*}
\item A string that ends in \bxshell{a} is represented by: \bxshell{*a}
\item A string that starts with \bxshell{a}, ends in \bxshell{b} and has unknown values (0 or more) in the middle is represented by: \bxshell{a*b}
\item A string which contains \bxshell{a} somewhere between other unknown characters (0 or more) is represented by: \bxshell{*a*}
\item To check for a text which begins with a star (\bxshell{*}), you must use the escape character: \verb+\\**+
\end{itemize}
 
