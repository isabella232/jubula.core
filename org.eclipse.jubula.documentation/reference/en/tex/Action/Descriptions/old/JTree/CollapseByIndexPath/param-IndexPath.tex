% CAP description for Tree --> Collapse by Indexpath --> Indexpath
\begin{itemize}
\item Use this parameter to specify the indexpath of the subtree you want to collapse.
\item Start at the top of the tree. The first node is \bxshell{1}, the second \bxshell{2} etc.
\item Use slash {\tt '/'} as a path separator (to separate parent nodes from child nodes).
\item For example, the second node within the first node is \bxshell{1/2}. 
\item Give the whole path to the node which you want to collapse.
\end{itemize}


\textbf{Example:}

\begin{itemize}
\item Your tree looks like this:

\begin{figure}
\begin{center}
\includegraphics{PS/Treeexample}
\caption{Tree 1}
\label{treeexample}
\end{center}
\end{figure}

\item You want to collapse node \bxcaption{C}. 
\item Enter \bxshell{1/1/2}:

\begin{figure}
\begin{center}
\includegraphics{PS/Treeexample2}
\caption{Tree 2}
\label{treeexample2}
\end{center}
\end{figure}

\item To collapse node \bxcaption{A}, enter \bxshell{1/1}:

\begin{figure}
\begin{center}
\includegraphics{PS/Treeexample3}
\caption{Tree 3}
\label{treeexample3}
\end{center}
\end{figure}

\item To collapse a whole tree, enter \bxshell{1}.
\end{itemize}
