\textbf{New function for accessing node attribute}
\begin{itemize}
\item A new function has been added to the set of pre-defined functions in \app{}.
\item The function is: \bxname{?getNodeAttribute()} and it can have either \bxname{name} or \bxname{comment} as arguments.
\item When \bxname{name} is chosen, the function reads the name of the node (e.g. \gdcase{}, \gdstep{}) on which it is resolved and uses this for the test.
\item  When \bxname{comment} is chosen, the function reads the comment of the node (e.g. \gdcase{}, \gdstep{}) on which it is resolved and uses this for the test.
\end{itemize}

\textbf{New function for accessing values in central data sets}
\begin{itemize}
\item There is a new function available in the pre-defined functions in \app{}.
\item The name of the new function is: \bxname{?getCentralDataSetValue()}, and it requires four arguments:
\begin{description}
\item [dataSetName:]{The name of the central data set to search in.}
\item [keyColumnName:]{The column that is used to find the correct line of the data set.}
\item [entryKey:]{The value to locate in the key column that will provide the correct line.}
\item [columnName:]{The name of the column in which the value to be chosen can be found.}
\end{description}
\item Within the specified data set, the required value is located based on the line found using the key column and entry key, as well as the column name for the actual value. 
\end{itemize}

\textbf{Multi-window support for HTML \gdauts{}}
\begin{itemize}
\item If your HTML \gdaut{} uses multi windows (e.g. pop-ups), then you can now specify this in the \gdaut{} configuration. 
\item \gdauts{} that are running in multi-window mode show the Selenium console as well as the \gdaut{} when the \gdaut{} is started. 
\item The \gdomeditor{} has a new button to allow switching between multiple open windows for mapping components, and there are new actions in the HTML unbound modules to allow you to switch between windows during the test. 
\end{itemize}

\textbf{On-click expansion for pictures in HTML reports}
\begin{itemize}
\item In HTML reports, you can now click on any screenshots taken on errors and they will be expanded to make the details more visible.
\item Clicking the image again will reduce it to its original size in the test result report.
\end{itemize}
