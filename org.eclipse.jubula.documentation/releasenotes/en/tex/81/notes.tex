\makeatletter
\section{Release Notes for \@bxversion}
\makeatother

\subsection{New Features and Developments}
\textbf{-datadir and -resultdir parameters are now optional for testexec}
\begin{itemize}
\item The parameters to enter a result directory (where HTML and XML reports are created) and to enter the place where any external data files reside are now optional in the testexec.
\item This reduces the amount of parameters you have to enter for a simple configuration.
\end{itemize}

\textbf{New optional parameter for testexec -resultname}
\begin{itemize}
\item You can now specify a name for the HTML and XML files that testexec can produce using the parameter -resultname.
\item If you do not use this parameter, then a default name consisting of the \gdsuite{} name will be used.
\end{itemize}

\textbf{Improved no-run option in testexec}
\begin{itemize}
\item The no-run option in the testexec now has a number of parameters so that you can define how far the test should be checked. 
\item You can use this option to perform all steps up to the actual test execution to ensure that e.g. \gdauts{} can be started, object mapping is complete etc.
\end{itemize}

\textbf{Composite components in JavaFX can now be mapped}
\begin{itemize}
\item It is now easier / possible to map composite components such as accordeons and choice boxes in JavaFX \gdauts{}. 
\end{itemize}

\textbf{Support for derived components in JavaFX}
\begin{itemize}
\item You can now test components in JavaFX \gdauts{} that are derived from currently supported JavaFX components. 
\end{itemize}

\textbf{New actions supported in iOS}
\begin{itemize}
\item You can now use the following actions in iOS tests: Application Input Text, Click in Component.
\end{itemize}

\textbf{Default columns changed in the Test Result Summary View}
\begin{itemize}
\item The columns that are shown by default in a new workspace for the Test Result Summary View have been reduced.
\item You can still add other columns to the view using the context menu in the view.
\item There is also a new button in the top right hand corner of the view to reset the standard column width.
\item There is a new column for the version number of the \gdproject{}.
\end{itemize}

\textbf{New actions in the iOS toolkit}
\begin{itemize}
\item The action \bxname{input text} is now available for iOS. You can use this action to enter text to a component that currently has focus.
\item The action \bxname{click in component} is now available for iOS.
\item The action \bxname{take screenshot} is now available for iOS. It takes a screenshot of the App screen.
\item There is a new action just for iOS \bxname{check badge value}. Use this action to check the text on a badge based on the index of that badge within a tab bar. 
\end{itemize}

\textbf{Dashboard as a deployable web application}
\begin{itemize}
\item You can now use the testresults.war found in the \bxname{dashboard/webapp} folder to deploy the dashboard into your own server.
\item The URL for the standalone and deployable dashboard has changed to unify it. You must update any links to the dashboard to use this new URL.
\begin{verbatim}
http://HOSTNAME:SERVERPORT/testresults/dashboard
\end{verbatim}

\end{itemize}

\textbf{New searching options}
\begin{itemize}
\item In the Search Dialog, you can now search for nodes in the \gdproject{} based on their name, their comment and/or their Task ID.
\end{itemize}

\subsection{Known issues and other information}
\textbf{Support for iOS 6 dropped}
\begin{itemize}
\item We no longer support iOS 6 for testing. The iOS support is now for 7 applications. 
\end{itemize}

\textbf{Relevance now a property of \gdsuites{}}
\begin{itemize}
\item Whether or not a test run is relevant is now determined by a property on each \gdsuite{}. This better reflects the situation that specific \gdsuites{} will always be relevant and others never. 
\item The meaning of relevance has not changed. However, the entry from the preferences has been removed and the parameter for the testexec is no longer accepted. 
\item All \gdsuites{} in newly imported \gdprojects{} are automatically set to true for relevance.
\item Newly created \gdsuites{} are created as being relevant.
\end{itemize}

\textbf{Dashboard URL has been changed}
\begin{itemize}
\item The dashboard URL has been changed due to 
\end{itemize}
