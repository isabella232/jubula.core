For SWT and RCP \gdauts{}, you need to define a keyboard layouts for the \gdaut{} (i.e. which keyboard layout is set for the system on which the \gdaut{} runs).

The  keyboard layouts for German (Germany) and English (US) are already present. If you want to use another keyboard layout, follow the instructions in the next section to create one.  

\begin{enumerate}
\item The name of the file must be in the following format:\\
keyboardmapping\_<language>\_<COUNTRY>.properties\\
e.g. for US English: keyboardmapping\_en\_US.properties

See the section in this document \bxpref{langcodes} for language codes.
\item In the file, enter the codes for all characters which require you to press a modifier key. 
\item The format for these codes is:\\
<Character>=<Modifier>+<Character without modifier>\\
e.g. for the @ character: @=shift+2
\item The following symbols must be escaped with a backslash:\\
\verb+! = : \ ,+\\
e.g. \verb?\!=shift+1? for !
\item There are pictures of various keyboards on the following website:\\
\bxname{http://www.uni-regensburg.de/EDV/Misc/KeyBoards/}
\item To add the keyboard layout file, you must create a fragment containing your newly added keyboard layout file(s). The host plugin for the fragment is:\\
\bxname{org.eclipse.jubula.client.core}\\
The path within the fragment must be:\\
\bxname{resources/keyboard\_mapping/<name>.properties}\\
\bxtipp{If you name the file with the locale code, then the keyboard layout will be displayed in plain text in the \gdaut{} configuration dialog.}


\end{enumerate}
