% Local IspellDict: english
% --------------------------------------------------------
% Extension Manual content
% copyright by BREDEX GmbH 2004
% --------------------------------------------------------

%\include{bxtypography}
\title{Extending \app{}}
\author*{\app{} Team}{}
%\bxbanner{../../../share/PS/splash}
\maketitle
% ----------------------------------------------------------------------
%% % $Id: version.tex 7776 2009-01-30 17:08:26Z alexandra $
%% % Local Variables:
%% % ispell-check-comments: nil
%% % Local IspellDict: american
%% % End:
%% % --------------------------------------------------------
%% % User documentation
%% % copyright by BREDEX GmbH 2005
%% % --------------------------------------------------------
%% % this command can be inserted multiple times
%% \gdhelpid{}
%% % 
%% \begin{bxdescription}
%% \end{bxdescription}
%% %
%% \begin{bxsteps}
%% % use the \item command for single steps
%% \end{bxsteps}
%% % change <FILE> to the same filename you are editing
%% \bxinput{Links/<FILE>}
%% %
%% % other usefull commands are
%% %   \bxhint{}        to create a hint
%% %   \bxwarn{}        to describe a warning
\index{Project!Version}
\index{Versioning Projects}

\begin{enumerate}
\item To create a different version of a \gdproject{}, select:\\
\bxmenu{Test}{Create new version}{}.
\item An automatic suggestion for the next version number is provided.  
\item You can accept this version number or enter a different one.  
\item Click \bxcaption{OK} to create the new version. 
\item The new version of the \gdproject{} becomes active in the client. 
\bxtipp{Any test result summaries for the \gdproject{} are not duplicated in the new version. Tests that ran for previous versions of the \gdproject{} are, however, still in the \gddb{} to be used for long term analysis.}
\end{enumerate}
\bxtipp{You can also create a new version using the dbtool \bxpref{TasksDBToolCreateVersion}}

\bxdocinfo{RELEASE}{BREDEX GmbH}{\today}{}
%bxdocinfo{Status}{Verfasser}{freigegeben am}{Verteilerliste?}
% set tocdepth to subsection level
\setcounter{tocdepth}{2}
\tableofcontents
% uncomment the following line to hide all \bxcomment's (i.e. for a
% release)
% FIXME MH: L2H doesn't like \renewcommand, find other solution
\renewcommand{\bxcomment}[2]{}%
% ----------------------------------------------------------------------
\clearpage
% ----------------------------------------------------------------------

\setcounter{secnumdepth}{2}

When developing graphical applications, it is often necessary or
convenient to alter or combine the functionality of existing toolkit
components, or even to write entirely new ones, as the requirements or
concept of the software may dictate. These new components
generally cannot be tested by \app{} ''out of the box'', as the
behavior of custom components cannot be predicted, or they may deviate
from established standards of ''look and feel''. In order to overcome this limitation, \app{} offers an extension
API, which you can use to allow \app{} to test your custom components.

The following sections describe the steps involved in extending
\app{}.

\clearpage

% BREDEX LaTeX Template
%  \documentclass is either ``bxreport'' or ``bxarticle''
%% %                 option is bxpaper
%% \documentclass{bxarticle}
%% % ----------------------------------------------------------------------
%% \begin{document}
%% \title{}
%% \author{}
%% % \author*{Hauptautor}{Liste der Nebenautoren}
%% \maketitle
%% % ----------------------------------------------------------------------
%% \bxversion{0.1}
%% %\bxdocinfo{STATUS}{freigegeben durch}{freigegeben am}{Verteilerliste}
%% \bxdocinfo{DRAFT}{}{}{}
%% % ----------------------------------------------------------------------

%% \end{document}
This manual provides important information about the more technical side of working with \gd. In the following sections, you will find:
\begin{itemize}
%\item a list of the keyboard shortcuts available in \gd \bxpref{scut}.
\item an introduction to using regular expressions as parameters \bxpref{regex}.
\item an introduction to using simple matches in your parameters \bxpref{simplematch}
\item a reference of all supported actions and their parameters \bxpref{actparam}.
\item details about the constants used to enter paths and list items \bxpref{constants}. 
\item a guide to using the abstract components offered by \gd \bxpref{overviewfam}.
\item information on relative paths in \gd{} \bxpref{relativepath}.
\item a list of the special characters in \gd{} \bxpref{specialchar}.
\item a table of language codes \bxpref{langcodes}.
\item information on keyboard layout files \bxpref{keyboardlayout}.
%\item a list of common icons used in \gd \bxpref{icons}.
\item instructions on how to remotely debug your \gdaut{} with \gd{} \bxpref{debugging}.
%\item \gd error messages \bxpref{errormsgs}.
\end{itemize}


\subsection{Supported RCP \gdauts{}}
 \gdauts{} written with the Swing GUI toolkit are supported according to the following points:


\begin{itemize}
\item The \gdaut{} is written using Java 1.5 or higher. 
\item It uses a Java SE (Standard Edition) JRE.
\bxwarn{\gdauts{} based on the NetBeans framework, and NetBeans itself, are currently not supported.}
\end{itemize}




\subsection{Setting up an RCP \gdaut{} for testing}
\index{RCP}
\index{AUT!RCP}
\index{RCP AUT's}
\index{RCP!Remote Control}

If you want to test a \bxname{Rich Client Platform} application, you
must first unzip our \bxcaption{RCP Remote Control} plugin into your
\gdaut{}. This can be done as follows:


\begin{enumerate}
\item Locate the \app{} installation directory.
\item Extract the content of the \bxname{rcp-support.zip} folder into the \bxname{plugins} directory for your RCP \gdaut{}.
\bxtipp{When you install a new version of \app{}, you must repeat these steps with the new RCP remote control plugin. We recommend starting your \gdaut{} once with \bxshell{-clean} to ensure that the new remote control plugin is used. }
\item RCP applications generally have a configuration/config.ini file which contains the parameter \bxname{osgi.bundles}. This parameter may need to be modified to allow the RCP remote control plugin to load on \gdaut{} startup. The \bxname{org.eclipse.update.configurator} plugin automatically loads all plugins found in the plugins directory, which means that the RCP remote control plugin should start with the \gdaut{} if \bxname{org.eclipse.update.configurator@3:start} is already defined in the \bxname{osgi.bundles} parameter. Otherwise, you may need to add  \bxname{org.eclipse.jubula.rc.rcp} to the end of the \bxname{osgi.bundles} parameter.
\end{enumerate}

If you do not follow the above steps, the \gdagent{} will not be able to
communicate with your \gdaut{}!




\subsection{Keyboard Layouts}
\index{Keyboard Layout}

For RCP \gdauts{}, a keyboard layout must be entered in the \gdaut{} configuration \bxpref{AdvancedAUTConfig}. German (DE) and English (US) are provided as standard keyboard layouts. 

If you require a different keyboard layout, you must create a mapping file for the new language and place it in the server/resources directory. You also have to put this mapping file into the RCP Remote Control plug-in of your \gdaut{}:\\


<RCPAUT>/plugins/org.eclipse.jubula.rc.rcp\_<Version>/resources




\subsection{Design for testability in RCP}
\index{Design for Testability!Swing}
\index{Swing!Set Name}
\index{Set Name}
\index{Swing!Design For Testability}

Although \jb{} can relocate components in the \gdaut{} even when they are not named by the developers, using the \bxname{setName} method for the current Swing component class certainly makes it easier to test \gdauts{}. Even if a whole area of the \gdaut{} has changed, the component will still be found based on this unique name. 


\subsection{Component name generation in RCP}
\label{RCPgenerate}
\gdhelpid{autSettingWizardPagePageContextId}{Defining an AUT}
\gdhelpid{autWizardPageGenerateNamesPageContextId}{Name Generation in RCP}
\gdhelpid{autSettingsPageContextId}{Adding/editing AUT's}
\index{RCP!Generate Technical Names}

RCP \gdauts{} often use wizards and standard dialogs. The components in these dialogs are not often named by developers, and are in different places on Windows, Linux and Mac systems. 

\app{} lets you decide if unique names should be generated for these components in your \gdaut{}, if no name has been given. You can configure this in the \gdaut{} settings \bxpref{Defineaut}. We recommend selecting this option, as it makes your tests more robust to any changes and also makes platform-independent testing possible in RCP \gdauts{}. 


\subsection{Best practices for testing RCP \gdauts{}}
\index{RCP!Best Practices}
\index{Best Practices!RCP}

One of the features of RCP \gdauts{} is that they generally remember the state of the \gdaut{} (position of views, which perspective was open) when the \gdaut{} is closed. In order to make tests as robust as possible, we recommend starting each use case with a module to reset the perspective to its defaults, and testing with this default perspective. 



\clearpage

\chapter{Issues}
% BREDEX LaTeX Template
%  \documentclass is either ``bxreport'' or ``bxarticle''
%% %                 option is bxpaper
%% \documentclass{bxarticle}
%% % ----------------------------------------------------------------------
%% \begin{document}
%% \title{}
%% \author{}
%% % \author*{Hauptautor}{Liste der Nebenautoren}
%% \maketitle
%% % ----------------------------------------------------------------------
%% \bxversion{0.1}
%% %\bxdocinfo{STATUS}{freigegeben durch}{freigegeben am}{Verteilerliste}
%% \bxdocinfo{DRAFT}{}{}{}
%% % ----------------------------------------------------------------------

%% \end{document}
This manual provides important information about the more technical side of working with \gd. In the following sections, you will find:
\begin{itemize}
%\item a list of the keyboard shortcuts available in \gd \bxpref{scut}.
\item an introduction to using regular expressions as parameters \bxpref{regex}.
\item an introduction to using simple matches in your parameters \bxpref{simplematch}
\item a reference of all supported actions and their parameters \bxpref{actparam}.
\item details about the constants used to enter paths and list items \bxpref{constants}. 
\item a guide to using the abstract components offered by \gd \bxpref{overviewfam}.
\item information on relative paths in \gd{} \bxpref{relativepath}.
\item a list of the special characters in \gd{} \bxpref{specialchar}.
\item a table of language codes \bxpref{langcodes}.
\item information on keyboard layout files \bxpref{keyboardlayout}.
%\item a list of common icons used in \gd \bxpref{icons}.
\item instructions on how to remotely debug your \gdaut{} with \gd{} \bxpref{debugging}.
%\item \gd error messages \bxpref{errormsgs}.
\end{itemize}


\subsection{Supported RCP \gdauts{}}
 \gdauts{} written with the Swing GUI toolkit are supported according to the following points:


\begin{itemize}
\item The \gdaut{} is written using Java 1.5 or higher. 
\item It uses a Java SE (Standard Edition) JRE.
\bxwarn{\gdauts{} based on the NetBeans framework, and NetBeans itself, are currently not supported.}
\end{itemize}




\subsection{Setting up an RCP \gdaut{} for testing}
\index{RCP}
\index{AUT!RCP}
\index{RCP AUT's}
\index{RCP!Remote Control}

If you want to test a \bxname{Rich Client Platform} application, you
must first unzip our \bxcaption{RCP Remote Control} plugin into your
\gdaut{}. This can be done as follows:


\begin{enumerate}
\item Locate the \app{} installation directory.
\item Extract the content of the \bxname{rcp-support.zip} folder into the \bxname{plugins} directory for your RCP \gdaut{}.
\bxtipp{When you install a new version of \app{}, you must repeat these steps with the new RCP remote control plugin. We recommend starting your \gdaut{} once with \bxshell{-clean} to ensure that the new remote control plugin is used. }
\item RCP applications generally have a configuration/config.ini file which contains the parameter \bxname{osgi.bundles}. This parameter may need to be modified to allow the RCP remote control plugin to load on \gdaut{} startup. The \bxname{org.eclipse.update.configurator} plugin automatically loads all plugins found in the plugins directory, which means that the RCP remote control plugin should start with the \gdaut{} if \bxname{org.eclipse.update.configurator@3:start} is already defined in the \bxname{osgi.bundles} parameter. Otherwise, you may need to add  \bxname{org.eclipse.jubula.rc.rcp} to the end of the \bxname{osgi.bundles} parameter.
\end{enumerate}

If you do not follow the above steps, the \gdagent{} will not be able to
communicate with your \gdaut{}!




\subsection{Keyboard Layouts}
\index{Keyboard Layout}

For RCP \gdauts{}, a keyboard layout must be entered in the \gdaut{} configuration \bxpref{AdvancedAUTConfig}. German (DE) and English (US) are provided as standard keyboard layouts. 

If you require a different keyboard layout, you must create a mapping file for the new language and place it in the server/resources directory. You also have to put this mapping file into the RCP Remote Control plug-in of your \gdaut{}:\\


<RCPAUT>/plugins/org.eclipse.jubula.rc.rcp\_<Version>/resources




\subsection{Design for testability in RCP}
\index{Design for Testability!Swing}
\index{Swing!Set Name}
\index{Set Name}
\index{Swing!Design For Testability}

Although \jb{} can relocate components in the \gdaut{} even when they are not named by the developers, using the \bxname{setName} method for the current Swing component class certainly makes it easier to test \gdauts{}. Even if a whole area of the \gdaut{} has changed, the component will still be found based on this unique name. 


\subsection{Component name generation in RCP}
\label{RCPgenerate}
\gdhelpid{autSettingWizardPagePageContextId}{Defining an AUT}
\gdhelpid{autWizardPageGenerateNamesPageContextId}{Name Generation in RCP}
\gdhelpid{autSettingsPageContextId}{Adding/editing AUT's}
\index{RCP!Generate Technical Names}

RCP \gdauts{} often use wizards and standard dialogs. The components in these dialogs are not often named by developers, and are in different places on Windows, Linux and Mac systems. 

\app{} lets you decide if unique names should be generated for these components in your \gdaut{}, if no name has been given. You can configure this in the \gdaut{} settings \bxpref{Defineaut}. We recommend selecting this option, as it makes your tests more robust to any changes and also makes platform-independent testing possible in RCP \gdauts{}. 


\subsection{Best practices for testing RCP \gdauts{}}
\index{RCP!Best Practices}
\index{Best Practices!RCP}

One of the features of RCP \gdauts{} is that they generally remember the state of the \gdaut{} (position of views, which perspective was open) when the \gdaut{} is closed. In order to make tests as robust as possible, we recommend starting each use case with a module to reset the perspective to its defaults, and testing with this default perspective. 



\clearpage

% BREDEX LaTeX Template
%  \documentclass is either ``bxreport'' or ``bxarticle''
%% %                 option is bxpaper
%% \documentclass{bxarticle}
%% % ----------------------------------------------------------------------
%% \begin{document}
%% \title{}
%% \author{}
%% % \author*{Hauptautor}{Liste der Nebenautoren}
%% \maketitle
%% % ----------------------------------------------------------------------
%% \bxversion{0.1}
%% %\bxdocinfo{STATUS}{freigegeben durch}{freigegeben am}{Verteilerliste}
%% \bxdocinfo{DRAFT}{}{}{}
%% % ----------------------------------------------------------------------

%% \end{document}
This manual provides important information about the more technical side of working with \gd. In the following sections, you will find:
\begin{itemize}
%\item a list of the keyboard shortcuts available in \gd \bxpref{scut}.
\item an introduction to using regular expressions as parameters \bxpref{regex}.
\item an introduction to using simple matches in your parameters \bxpref{simplematch}
\item a reference of all supported actions and their parameters \bxpref{actparam}.
\item details about the constants used to enter paths and list items \bxpref{constants}. 
\item a guide to using the abstract components offered by \gd \bxpref{overviewfam}.
\item information on relative paths in \gd{} \bxpref{relativepath}.
\item a list of the special characters in \gd{} \bxpref{specialchar}.
\item a table of language codes \bxpref{langcodes}.
\item information on keyboard layout files \bxpref{keyboardlayout}.
%\item a list of common icons used in \gd \bxpref{icons}.
\item instructions on how to remotely debug your \gdaut{} with \gd{} \bxpref{debugging}.
%\item \gd error messages \bxpref{errormsgs}.
\end{itemize}


\subsection{Supported RCP \gdauts{}}
 \gdauts{} written with the Swing GUI toolkit are supported according to the following points:


\begin{itemize}
\item The \gdaut{} is written using Java 1.5 or higher. 
\item It uses a Java SE (Standard Edition) JRE.
\bxwarn{\gdauts{} based on the NetBeans framework, and NetBeans itself, are currently not supported.}
\end{itemize}




\subsection{Setting up an RCP \gdaut{} for testing}
\index{RCP}
\index{AUT!RCP}
\index{RCP AUT's}
\index{RCP!Remote Control}

If you want to test a \bxname{Rich Client Platform} application, you
must first unzip our \bxcaption{RCP Remote Control} plugin into your
\gdaut{}. This can be done as follows:


\begin{enumerate}
\item Locate the \app{} installation directory.
\item Extract the content of the \bxname{rcp-support.zip} folder into the \bxname{plugins} directory for your RCP \gdaut{}.
\bxtipp{When you install a new version of \app{}, you must repeat these steps with the new RCP remote control plugin. We recommend starting your \gdaut{} once with \bxshell{-clean} to ensure that the new remote control plugin is used. }
\item RCP applications generally have a configuration/config.ini file which contains the parameter \bxname{osgi.bundles}. This parameter may need to be modified to allow the RCP remote control plugin to load on \gdaut{} startup. The \bxname{org.eclipse.update.configurator} plugin automatically loads all plugins found in the plugins directory, which means that the RCP remote control plugin should start with the \gdaut{} if \bxname{org.eclipse.update.configurator@3:start} is already defined in the \bxname{osgi.bundles} parameter. Otherwise, you may need to add  \bxname{org.eclipse.jubula.rc.rcp} to the end of the \bxname{osgi.bundles} parameter.
\end{enumerate}

If you do not follow the above steps, the \gdagent{} will not be able to
communicate with your \gdaut{}!




\subsection{Keyboard Layouts}
\index{Keyboard Layout}

For RCP \gdauts{}, a keyboard layout must be entered in the \gdaut{} configuration \bxpref{AdvancedAUTConfig}. German (DE) and English (US) are provided as standard keyboard layouts. 

If you require a different keyboard layout, you must create a mapping file for the new language and place it in the server/resources directory. You also have to put this mapping file into the RCP Remote Control plug-in of your \gdaut{}:\\


<RCPAUT>/plugins/org.eclipse.jubula.rc.rcp\_<Version>/resources




\subsection{Design for testability in RCP}
\index{Design for Testability!Swing}
\index{Swing!Set Name}
\index{Set Name}
\index{Swing!Design For Testability}

Although \jb{} can relocate components in the \gdaut{} even when they are not named by the developers, using the \bxname{setName} method for the current Swing component class certainly makes it easier to test \gdauts{}. Even if a whole area of the \gdaut{} has changed, the component will still be found based on this unique name. 


\subsection{Component name generation in RCP}
\label{RCPgenerate}
\gdhelpid{autSettingWizardPagePageContextId}{Defining an AUT}
\gdhelpid{autWizardPageGenerateNamesPageContextId}{Name Generation in RCP}
\gdhelpid{autSettingsPageContextId}{Adding/editing AUT's}
\index{RCP!Generate Technical Names}

RCP \gdauts{} often use wizards and standard dialogs. The components in these dialogs are not often named by developers, and are in different places on Windows, Linux and Mac systems. 

\app{} lets you decide if unique names should be generated for these components in your \gdaut{}, if no name has been given. You can configure this in the \gdaut{} settings \bxpref{Defineaut}. We recommend selecting this option, as it makes your tests more robust to any changes and also makes platform-independent testing possible in RCP \gdauts{}. 


\subsection{Best practices for testing RCP \gdauts{}}
\index{RCP!Best Practices}
\index{Best Practices!RCP}

One of the features of RCP \gdauts{} is that they generally remember the state of the \gdaut{} (position of views, which perspective was open) when the \gdaut{} is closed. In order to make tests as robust as possible, we recommend starting each use case with a module to reset the perspective to its defaults, and testing with this default perspective. 



